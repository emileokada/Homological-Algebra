\documentclass{memoir}

\input{header}
\title{Homological algebra}
\author{Emile T. Okada}

\begin{document}
\maketitle
\tableofcontents
\chapter{Abelian Categories}
\section{Additive categories}
Let $\mathcal C$ be a category such that the hom-sets carry the structure of an abelian group and composition is bilinear. We call such a category $\Ab$-enriched.
An additive category is an $\Ab$-enriched category which has finite coproducts.
\section{Abelian categories}
\section{Exact sequences}
\label{sec:es}
\begin{equation}
    \begin{tikzcd}
        & & \coker(\theta) \\
        \mathcal F \arrow[r,"\theta"] \arrow[d] & \mathcal G \arrow[r,"\phi"] \arrow[ur] & \mathcal H \\
        \im(\theta) \arrow[r,"\alpha"] \arrow[ur] & \ker(\phi) \arrow[u] &
    \end{tikzcd}
\end{equation}
\subsection{Split sequences}
\subsection{Homology}
\section{Adjoint functors}
Let $L:\mathcal A\to\mathcal B$ be an additive functor between abelian categories.
If $L$ admits a right adjoint $R:\mathcal B\to\mathcal A$ then it turns out $L$ has a lot of useful properties. 
In this section we explore these properties.
\begin{proposition}
    Suppose $L\dashv R$. Then $L$ is right exact and $R$ is left exact.
\end{proposition}
\begin{proof}
    Consider the short exact sequence $0\to B_1\to B_2\to B_3\to 0$. For every $A\in\mathcal A$ we get the following commutative diagram 
    \begin{equation}
        \begin{tikzcd}
            0 \arrow[r] & \Hom(L(A),B_1) \arrow[r] \arrow[d,"\cong"] & \Hom(L(A),B_2) \arrow[r] \arrow[d,"\cong"] & \Hom(L(A),B_3) \arrow[d,"\cong"] \\
            0 \arrow[r] & \Hom(A,R(B_1)) \arrow[r] & \Hom(A,R(B_2)) \arrow[r] & \Hom(A,R(B_3))
        \end{tikzcd}
    \end{equation}
    where the top row is exact.
    It follows that the bottom row is exact for all $A$ and so the bottow row is too.
    It follows that 
    \begin{equation}
        \begin{tikzcd}
            0 \arrow[r] & R(B_1) \arrow[r] & R(B_2) \arrow[r] & R(B_3)
        \end{tikzcd}
    \end{equation}
    is exact and so $R$ is left exact.
    By a similar argument $L$ is right exact.
\end{proof}
\begin{proposition}
    Suppose $L\dashv R$. Then
    \begin{enumerate}
        \item if $L$ is exact then $R$ preserves injectives
        \item if $R$ is exact then $L$ preserves projectives.
    \end{enumerate}
\end{proposition}
\begin{proof}
    Suppose $L$ is exact and $I$ is an injective object in $\mathcal B$.
    We need to show that $\Hom(-,R(I))$ is exact.
    To do this it suffices to show that given $f:A\to B$ injective, the map $f^*:\Hom(B,R(I))\to\Hom(A,R(I))$ is surjective.
    But $L$ is exact so $Lf$ is injective and so $(Lf)^*:\Hom(LB,I)\to\Hom(LA,I)$ is surjective.
    We also have that $L\dashv R$ and so 
    \begin{equation}
        \begin{tikzcd}
            \Hom(L(B),I) \arrow[d,"\cong"] \arrow[r,"(Lf)^*"] & \Hom(L(A),I) \arrow[d,"\cong"] \\
            \Hom(B,R(I)) \arrow[r,"f^*"] & \Hom(A,R(I))
        \end{tikzcd}
    \end{equation}
    commutes.
    It follows that $f^*$ is surjective as required.

    The corresponding result for $R$ follows similarly.
\end{proof}
\chapter{Sheaf Theory}
\section{Presheaves}
Let $\mathcal C$ be any category, $\mathcal A$ be an abelian category and define $\PreSh(\mathcal C) = \Fun(\mathcal C^{op},\mathcal A)$ to be the category of presheaves on $\mathcal C$ with values in $\mathcal A$.
The functor sending all objects to $0$ is certainly both initial and terminal, direct sums can be defined pointwise, and the hom-sets in $\PreSh(\mathcal C)$ inherit an additive structure from $\mathcal A$ so $\PreSh(\mathcal C)$ is naturally an additive category.
Moreover kernels and cokernels can be contructed in the obvious way and it is clear that they satisfy the axioms for an abelian category and so $\PreSh(C)$ is abelian.
\section{Sheaves}
To define sheaves we restrict to the case when $X$ be a topological space, $\mathcal U$ the poset of open sets of X, and $\mathcal A$ be an abelian category.
We write $\PreSh(X)$ for $\PreSh(\mathcal U)$.
The category of sheaves on $X$ with values in $\mathcal A$, $\Sh(X)$, is defined to be the full subcategory of $\PreSh(X)$ with objects given by presheaves $\shf{F}$ for which the following diagram is an equalizer for all open coverings $U = \cup_i U_i$
\begin{equation}
\shf{F}(U)\rightarrow \prod_i \shf{F}(U_i) \rightrightarrows \prod_{i,j} \shf{F}(U_i\cap U_j).
\end{equation}
Since $\mathcal A$ is an abelian category this is equivalent to the following diagram being exact
\begin{equation}
0\rightarrow \shf{F}(U)\rightarrow \prod_i \shf{F}(U_i) \xrightarrow[]{\text{diff}} \prod_{i,j} \shf{F}(U_i\cap U_j).
\end{equation}
Note that since $\varnothing$ admits the empty covering and the empty product is $0$ this forces $\shf{F}(\varnothing) = 0$.

As in the case of $\PreSh(\mathcal C)$, $\Sh(X)$ is an additive category.
However, the cokernel of a morphism between sheaves need not be a sheaf and so we must do some more work to show that $\Sh(X)$ is abelian.

Fix $x\in X$.
For a (pre)sheaf $\shf{F}$ define the stalk of $\shf{F}$ at $x$ to be 
\begin{equation}
    \shf{F}_x = \varinjlim_{U\ni x}\shf{F}
\end{equation}
when this limit exists. Note that this is a functor since morphisms between (pre)sheaves are natural transformations.
\begin{thm}
    Let $\phi:\shf{F}\to \shf{G}$ be a morphism of sheaves.
    \begin{enumerate}
        \item If $\phi_x$ is injective for all $x\in X$ then $\phi$ is injective on sections.
        \item If $\phi_x$ is an isomorphism for all $x\in X$ then $\phi$ is an isomorphism.
    \end{enumerate}
\end{thm}
\begin{proof}
Exercise.
\end{proof}
\subsubsection{Aside}
Although we do not need this right away, given an $A\in \mathcal A$ we can define the (pre)sheaf $x_*A$ by 
\begin{equation}
    (x_*A)(U) = \begin{cases}
        A & \textit{if } x\in U \\
        0 & otherwise
    \end{cases}
\end{equation}
\begin{proposition}
    When it exists, the functor $(-)_x:\Sh(X)\to \mathcal A$ is left adjoint to $x_*:\mathcal A\to \Sh(X)$.
\end{proposition}
\begin{proof}
    To see this simply note that morphisms between $\shf{F}$ and $x_*(A)$ correspond naturally to natrual transformations between $\shf{F}$ restricted to $U\ni x$ and $\Delta(A)$.
\end{proof}
\begin{remark}
    The result also holds in $\PreSh(X)$.
\end{remark}
\section{\'Etal\'e space of a presheaf and sheafification}
For a presheaf $\shf{F}$ we are now in the position to define its \'etal\'e space.
The \'etal\'e space of $\shf{F}$, denoted $\Spe(\shf{F})$ is the topological space with underlying set $\amalg_{x\in X}\shf{F}_x$ and topology generated by the basis of sets given by $\{s_x|x\in U\}$ for $s\in \shf{F}(U)$ where $U\subset X$ is open.
Together with this space there is also a natural continuous map $\pi:\Spe(\shf{F})\to X$ sending an element $s_x$ to $x$.
The sheafification of $\shf{F}$, denoted $\shf{F}^+$, is then defined to be the sheaf of sections of $\pi:\Spe(\shf{F})\to X$.
By unwrapping the definitions we see that the sections can be characterised as
\begin{align}
    \shf{F}^+(U) = \{s:U\to\amalg_{x\in U}\shf{F}_x : & \forall x\in U, \exists V\subset U \text{ open containing $x$ and } \nonumber \\
                                                            & t\in\shf{F}(V) \text{ s.t. }  s(y) = t_y \forall y\in V\}
\end{align}
In particular there is a natural morphism $\shf{F} \to \shf{F}^+$ sending $s\in \shf{F}(U)$ to the section $x\mapsto s_x$ which is an isomorphism on stalks.
From the characterisation of sections it clear that if $\shf{F}$ is a presheaf of $\AbGrp,\Ring,...$ then $\shf{F}^+$ is a sheaf with values in the corresponding abelian category.

We have defined $\Spe$ and $(-)^+$ on objects but they can also be turned into functors.
If we have a morphism $\phi:\shf{F}\to\shf{G}$ between presheaves, this induces a continuous map $\Spe(\phi):\Spe(\shf{F})\to \Spe(\shf{G})$ given by $s_x\mapsto\phi_x(s_x)$ so that 
\begin{equation}
    \begin{tikzcd}
        \Spe(\shf{F}) \arrow[rr, "\Spe(\phi)"] \arrow[rd,"\pi"] & & \Spe(\shf{G}) \arrow[dl,"\pi"] \\
                                              & X &
    \end{tikzcd}
\end{equation}
commutes.
This construction is functorial and turns $\Spe$ into a functor from presheaves to topological bundles over $X$.
It follows that we also obtain a map of sheaves $\phi^+:\shf{F}^+\to\shf{G}^+$ by composing sections with $\Spe(\phi)$.
Thus we have a functor $(-)^+:\PreSh(X)\to\Sh(X)$ and in fact the following diagram commutes.
\begin{equation}
    \label{eq:sheafif}
    \begin{tikzcd}
        \shf{F}^+ \arrow[r,"\phi^+"] & \shf{G}^+ \\
        \shf{F} \arrow[u] \arrow[r,"\phi"] & \shf{G} \arrow[u]
    \end{tikzcd}
\end{equation}
Note that since the morphism $\shf{F}\to\shf{F}^+$ is an isomorphism when $\shf{F}$ is a sheaf, this says that the functor $(-)^+$ restricted to $\Sh(X)$ is natrually isomorphism to the identity functor.

\begin{thm}
    Let $\theta:\shf{F}\to\shf{F}^+$ be the natural morphism. 
    Then for any morphism of presheaves $\phi:\shf{F}\to \shf{G}$ with $\shf{G}$ a sheaf, there exists a unique morphism of sheaves $\psi:\shf{F}^+\to\shf{G}$ so that 
    \begin{equation}
        \begin{tikzcd}
            \shf{F}^+ \arrow[r,"\psi"] & \shf{G} \\
            \shf{F} \arrow[ur,"\phi"] \arrow[u,"\theta"]
        \end{tikzcd}
    \end{equation}
    commutes.
\end{thm}
\begin{proof}
    This just follows from equation \ref{eq:sheafif}, the fact that $\theta:\shf{G}\to\shf{G}^+$ is an isomorphism when $\shf{G}$ is a sheaf, and by taking stalks.
\end{proof}
\begin{corollary}
    The sheafification functor is left adjoint to the inclusion functor $\iota:\Sh(X)\to\PreSh(X)$.
\end{corollary}
\begin{proof}
    Let $\shf{F}$ be a presheaf and $\shf{G}$ be a sheaf.
    Given a morphism $\phi:\shf{F}^+\to\shf{G}$ we can precompose it with $\theta:\shf{F}\to\shf{F}^+$ to obtain a map $\shf{F}\to\iota\shf{G}$.
    Conversely, given $\psi:\shf{F}\to\iota\shf{G}$, we obtain a map $\shf{F}^+\to \shf{G}$ from the theorem.
    Then the theorem says these operations are inverse so we have a bijection 
    \begin{equation}
        \Hom(\shf{F}^+,\shf{G}) \cong \Hom(\shf{F},\iota\shf{G}).
    \end{equation}
    Naturality is then an easy check.
\end{proof}
\begin{corollary}
    The sheafification functor is exact.
\end{corollary}
\begin{proof}
    It is a left adjoint so it is right exact.
    It thus suffices to show that if $\phi:\shf{F}\to\shf{G}$ is injective then so is $\phi^+$.
    For this it suffices to show that $\phi_x$ is injective for all $x$.
    But this is obvious.
\end{proof}
We can now define the cokernel of a morphism $\phi:\shf{F}\to\shf{G}$ in $\Sh(X)$.
We simply define it to be the sheafification of the cokernel in $\PreSh(X)$ and it is an easy to check to see that this is indeed a cokernel object in $\Sh(X)$.
It is then easy to see that $\ker\coker = \coker\ker$ by looking at stalks and so $\Sh(X)$ is an abelian category.
\begin{remark}
    While $\Sh(X)$ is a full subcategory of $\PreSh(X)$ that is abelian, it is not a full abelian subcategory.
\end{remark}
\section{Exact sequences}
Now that we know that we are working in an abelian category we can talk about exact sequences in $\Sh(X)$.
Recall from section \ref{sec:es} that $\shf{F}\xrightarrow{\theta}\shf{G}\xrightarrow{\phi}\shf{H}$ is exact at $\shf{G}$ if $\phi\circ\theta=0$ and the map induced map $\im(\theta)\to\ker(\phi)$ is an isomorphism.
But the map $\im(\theta)\to\ker(\phi)$ is an isomorphism iff it is an isomorphism at the level of stalks iff $\shf{F}_x\xrightarrow{\theta_x}\shf{G}_x\xrightarrow{\phi_x}\shf{H}_x$ is exact for all $x\in X$.
Thus exactness in $\Sh(X)$ can be verified by checking exactness at all the stalks.
\section{Sheaves over different spaces}
\subsection{Direct image sheaf}
Let $f:X\to Y$ be a continuous map between topological spaces and $\shf{F}$ a sheaf on $X$.
We define the direct image of $\shf{F}$ under $f$ to be the sheaf $f_*\shf{F}$ on $Y$ defined by $f_*\shf{F}(U) = \shf{F}(f^{-1}(U))$.
If we define $f_*$ on morphisms in the obvious way then it is clear that we obtain a functor $f_*:\Sh(X)\to\Sh(Y)$.
In fact we also obtain a functor $f_*:\PreSh(X)\to\PreSh(Y)$ and it turns out this functor has nice left adjoint.

Define $\lim_f:\PreSh(Y)\to\PreSh(X)$ to be the functor that sends $\shf{F}\in \PreSh(Y)$ to the presheaf $\lim_f(\shf{F})(U) = \varinjlim_{V\supset f(U)}\shf{F}(V)$ on $X$, and does the obvious things to morphisms.
\begin{thm}
    $\lim_f\dashv f_*$ as functors between $\PreSh(X)$ and $\PreSh(Y)$.
\end{thm}
\begin{proof}
    Let $\phi:\lim_f\shf{F}\to \shf{G}$ be a morphism of presheaves.
    For $V$ open in $Y$, $f^{-1}(V)$ is open in $X$ and so we have maps
    \begin{equation}
        \shf{F}(V)\to\varinjlim_{W\supset f(U)}\shf{F}(W) \to \shf{G}(U)
    \end{equation}
    where $U=f^{-1}(V)$.
    If $V'\subset V$, $U = f^{-1}(V)$ and $U'=f^{-1}(V')$ then
    \begin{equation}
        \begin{tikzcd}
            \shf{F}(V) \arrow[r] \arrow[d] \arrow[dr] & \displaystyle\varinjlim_{W\supset f(U)}\shf{F}(W) \arrow[r] \arrow[d] & \shf{G}(U) \arrow[d] \\
            \shf{F}(V') \arrow[r] & \displaystyle\varinjlim_{W\supset f(U')}\shf{F}(W) \arrow[r] & \shf{G}(U')
        \end{tikzcd}
    \end{equation}
    commutes and so these maps in fact define a morphism $\shf{F}\to f_*\shf{G}$.

    Conversely suppose we are given a morphism $\shf{F}\to f_*\shf{G}$.
    Let $U$ be open in $X$.
    For $V\supset f(U)$ we have maps
    \begin{equation}
        \shf{F}(V)\to \shf{G}(f^{-1}(V)) \to \shf{G}(U).
    \end{equation}
    Moreover if $V\supset V'\supset f(U)$ then
    \begin{equation}
        \begin{tikzcd}[column sep=tiny,row sep=tiny]
            \shf{F}(V) \arrow[r] \arrow[dd] & \shf{G}(f^{-1}(V)) \arrow[dd] \arrow[dr] & \\ 
                                 & & \shf{G}(U) \\
            \shf{F}(V') \arrow[r] & \shf{G}(f^{-1}(V')) \arrow[ur] & \\ 
        \end{tikzcd}
    \end{equation}
    commutes so we obtain maps $\varinjlim_{V\supset f(U)}\shf{F}(V)\to\shf{G}(U)$.
    If $U\supset U'$ we have maps
    \begin{equation}
        \begin{tikzcd}
            \displaystyle \varinjlim_{V\supset f(U)}\shf{F}(V) \arrow[r] \arrow[d] & \shf{G}(U) \arrow[d] \\ 
            \displaystyle \varinjlim_{V\supset f(U')}\shf{F}(V) \arrow[r] & \shf{G}(U').
        \end{tikzcd}
    \end{equation}
    A straighforward calculation shows that this commutes and so we obatin a morphism $\lim_f\shf{F}\to\shf{G}$.

    These operations are clearly inverse to each other.
    A straightforward calculation shows that the bijection is natural.
\end{proof}
\begin{corollary}
    $\lim_f$ is an exact functor.
\end{corollary}
\begin{proof}
    It is a left adjoint so it is right exact. 
    Thus it suffices to show that it sends injective maps to injective maps. But this is obvious.
\end{proof}
\subsection{Inverse image sheaf}
Let $f:X\to Y$ be a continuous map between topological spaces and $\shf{F}$ a sheaf on $Y$.
Let $f^{-1}\Spe(\shf{F})$ be the pullback 
\begin{equation}
    \begin{tikzcd}
        f^{-1}\Spe(\shf{F}) \arrow[r,dashed] \arrow[d,dashed,"\pi"] \arrow[dr, phantom, "\lrcorner", very near start] & \Spe(\shf{F}) \arrow[d,"\pi"] \\
        X \arrow[r,"f"] & Y.
    \end{tikzcd}
\end{equation}
We define the inverse image sheaf $f^{-1}\shf{F}$ to be the sheaf of sections of $\pi:f^{-1}\Spe(\shf{F})\to X$.
Equivalently, it is the sheaf
\begin{equation}
    \label{eq:invimg}
    f^{-1}\shf{F}(U) = \left\{s:U\to\Spe(\shf{F}): 
    \begin{tikzcd}
        & \Spe(\shf{F}) \arrow[d,"\pi"] \\
        U \arrow[r,"f|_U"] \arrow[ur,"s"] & Y
    \end{tikzcd} 
    \text{ commutes }
    \right\}
\end{equation}
or also equivalently, the sheaf
\begin{align}
    f^{-1}\shf{F}(U) = \{s:U\to\amalg_{x\in U}\shf{F}_{f(x)} : & \forall x\in U, \exists W\subset Y, V\subset f^{-1}(W)\cap U \text{ open and} \nonumber \\
                                                             & t\in\shf{F}(W) \text{ s.t. } x\in V \wedge  s(y) = t_{f(y)} \forall y\in V\}.
\end{align}
It is clear from the construction that we obtain a functor $f^{-1}:\Sh(Y)\to\Sh(X)$.
\begin{remark}
    A direct calculation shows that $f^{-1}\shf{F}_x$ and $\shf{F}_{f(x)}$ are naturally isomorphic and so there is a natrual bijection between $f^{-1}\Spe(\shf{F})$ and $\Spe(f^{-1}\shf{F})$.
    It is then a straightforward exercise to check that this bijection is in fact a homeomorphism i.e. $f^{-1}\Spe(\shf{F})\cong\Spe(f^{-1}\shf{F})$.
\end{remark}
\begin{thm}
    $f^{-1}$ is naturally isomorphic to $(-)^+\circ \lim_f$.
\end{thm}
\begin{proof}
    Let $U$ be an open subset of $X$ and $s\in\lim_f\shf{F}(U)$.
    There is a natural map $\phi_x:(\lim_f\shf{F})_x\to\shf{F}_{f(x)}$ so we can define a map $U\to\Spe(\shf{F})$ by $x\mapsto \phi_x(s_x)$.
    It is clear that this gives an element of $f^{-1}\shf{F}(U)$ as characterised by equation \ref{eq:invimg}.
    Thus we obtain a morphism $\lim_f\shf{F}\to f^{-1}\shf{F}$.
    On stalks this map is given by $\phi_x$.
    A direct calculation shows that $\phi_x$ is an isomorphism for all $x\in X$ and so the induced map $(\lim_f\shf{F})^+\to f^{-1}\shf{F}$ must be an isomorphism.
    It is straightforward to see that this defines a natural transformation.
\end{proof}
\begin{corollary}
    $f^{-1}\dashv f_*$ as functors between $\Sh(X)$ and $\Sh(Y)$.
\end{corollary}
\begin{proof}
    $f^{-1}$ is naturally isomophic to $(-)^+\circ \lim_f$ and so for $\shf{F}\in\Sh(Y),\shf{G}\in\Sh(X)$ we have natural bijections
    \begin{align}
        \Hom_{\Sh(X)}(f^{-1}\shf{F},\shf{G}) &\cong \Hom_{\Sh(X)}\left((\lim_f\shf{F})^+,\shf{G}\right) \cong \Hom_{\PreSh(X)}\left(\lim_f\shf{F},\shf{G}\right) \nonumber \\
                                             &\cong \Hom_{\PreSh(Y)}\left(\shf{F},f_*\shf{G}\right) \cong \Hom_{\Sh(Y)}\left(\shf{F},f_*\shf{G}\right).
    \end{align}
\end{proof}
\begin{corollary}
    $(-)_x\circ f^{-1} = (-)_{f(x)}$.
\end{corollary}
\begin{proof}
    $(-)_x\circ f^{-1} = (-)_x \circ (-)^+ \circ \lim_f = (-)_x \circ \lim_f = (-)_{f(x)}$.
\end{proof}
\begin{corollary}
    $f^{-1}$ is an exact functor.
\end{corollary}
\begin{proof}
    It is the composition of two exact functors.
    Alternatively take stalks.
\end{proof}
\section{The $\sheafHom$ sheaf}
Let $\shf{F}$ and $\shf{G}$ be sheaves and $f:\Spe(\shf{F})\to \Spe(\shf{G})$ be a continuous map so that 
\begin{equation}
    \begin{tikzcd}
        \Spe(\shf{F}) \arrow[rr, "\Spe(\phi)"] \arrow[rd,"\pi"] & & \Spe(\shf{G}) \arrow[dl,"\pi"] \\
                                              & X &
    \end{tikzcd}
\end{equation}
commutes.
Then we obtain a morphism $\shf{F}^+\to\shf{G}^+$ by postcomposing sections with $f$.
Since $\shf{F}$ and $\shf{G}$ are sheaves we in fact obtain a morphism $\shf{F}\to\shf{G}$.
But we also know that morphisms $\shf{F}\to \shf{G}$ give continuous maps $\Spe(\shf{F})\to\Spe(\shf{G})$ making the above diagram commute.
\section{Injective sheaves}
There are enough injectives.
\chapter{Spectral sequences}

\chapter{Group cohomology}
\end{document}

